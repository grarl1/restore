\documentclass[runningheads]{llncs}

% Full page margins
\usepackage{fullpage}

% For language
\usepackage[english]{babel}
\usepackage[utf8]{inputenc}

% Glossary
\usepackage[acronym]{glossaries}
\makenoidxglossaries

% Acronyms
\newacronym{dl}{DL}{Deep learning}
\newacronym{relu}{ReLU}{Rectified Linear Unit}
\newacronym{ann}{ANN}{artificial neural network}
\newacronym{sisr}{SISR}{single image super-resolution}
\newacronym{hr}{HR}{high resolution}
\newacronym{lr}{LR}{low resolution}
\newacronym{cnn}{CNN}{convolutional neural network}
\newacronym{srcnn}{SRCNN}{Super-Resolution Convolutional Neural Network}
\newacronym{fsrcnn}{FSRCNN}{Fast Super-Resolution Convolutional Neural Network}
\newacronym{ircnn}{IRCNN}{Image Restoration Convolutional Neural Network}
\newacronym{psnr}{PSNR}{Peak Signal to Noise Ratio}
\newacronym{ssim}{SSIM}{structural similarity}
\newacronym{gpu}{GPU}{graphics processing unit}

% For figures
\usepackage{graphicx}

% Standard colors
\usepackage[dvipsnames]{xcolor}

% For links
\usepackage{hyperref}
\usepackage{cleveref}
\hypersetup {
	linkcolor  = MidnightBlue,
	citecolor  = MidnightBlue,
	urlcolor   = MidnightBlue,
	colorlinks = true,
}

% For bibliography
\usepackage[backend=biber]{biblatex}
\usepackage{csquotes}
\addbibresource{references.bib}

\begin{document}

	% Title
	\title{Single Image Super-Resolution and Denoising using Convolutional Neural Networks}
	\author{Guillermo Ruiz Álvarez}
	\institute{University of Málaga - \email{grabmct@uma.es}}
	
	\maketitle
	
	\abstract{We analyze the behavior of applying two well-known deep learning methods for single image super-resolution and image denoising, namely FSRCNN \cite{FSRCNN} and IRCNN \cite{IRCNN}, in different order to evaluate their combined capability to restore low resolution images with presence of noise. In order not to limit the case study to a single type of noise, Gaussian noise, Poisson noise, salt-and-pepper noise and uniform noise are applied to down-scaled images in order to construct the degraded images to be restored. As a result, we find that applying IRCNN before super-resolving the images with FSRCNN shows better results than using the networks in the opposite order for all types of noise. We also compare the quality achieved by using these deep learning methods in combination with other traditional methods. Experimental results show that FSRCNN and IRCNN achieve better restoration quality than when we use bicubic interpolation for super-resolution and median or wavelet filters for image denoising.
	
	\keywords{Single Image Super-Resolution, Image Denoising, Image Restoration, Convolutional Neural Networks.}\\~
	
	\textbf{Supervisors:} Ezequiel López Rubio, Rafael Marcos Luque Baena.
	
	\section{Introduction}

\Gls{sisr} and image denoising are two common low-level tasks in the field of computer vision. \Gls{sisr} is a classical problem that aims at recovering a \gls{hr} image from a given \gls{lr} version of the same image. This problem is ill-posed because a given \gls{lr} input can correspond to multiple \gls{hr} solutions and therefore reliable prior information is usually required to constraint the solution space \cite{DBLP:SISR} \cite{SRCNN} \cite{DBLP:DEEPSISR}. On the other hand, image denoising is the process of estimating the clean and original version of an image given a noisy input.

\gls{sisr} and image denoising have received increasing attention by researches due to their multiple applications in the field of computer vision. The various methods that have been developed to tackle these two tasks can be broadly classified into two groups: traditional and deep learning methods. Deep learning methods, and more specifically those using \glspl{cnn}, have achieved promising results in terms of performance and restoration of quality for both tasks \cite{DBLP:DEEPNR} \cite{DBLP:DEEPSISR}.

However, applying \gls{sisr} and denoising methods in conjunction might produce undesired results if the presence of noise in the \gls{lr} images affects the super-resolution task or, likewise, if the result of denoising an input image degrades the quality of the super-resolution step. Therefore, this study aims at analyzing which is the most efficient process to produce \gls{hr} images given \gls{lr} inputs with presence of different types of noise using deep learning methods based on \glspl{cnn}. For this study, we have selected Gaussian noise, Poisson noise, salt-and-pepper noise and uniform noise to be applied to down-scaled versions of the \gls{hr} images.

Among all the deep learning methods for \gls{sisr} and image denoising, experimental results using the \gls{fsrcnn} \cite{FSRCNN} and the \gls{ircnn} \cite{IRCNN} have demonstrated promising results and high performance in their implementations in \glspl{gpu} and therefore we have selected them in order to perform the case study.

For the analysis, both networks have been trained using the same training and validation datasets (BSD200 \cite{BSDS}, General100 \cite{FSRCNN} and T91 \cite{T91}) and have been evaluated with Set5 \cite{SET5} and Set14 \cite{SET14} using \gls{psnr} and \gls{ssim} as quality metrics.

Furthermore, we have also selected traditional methods for both super-resolution and image denoising in order to compare the efficiency of applying traditional and deep learning methods to tackle these low-level computer vision tasks. More specifically, we have used bicubic interpolation for \gls{sisr} and median and wavelet filtering for image denoising.

The contribution of this study can be summarized as follows:
\begin{itemize}
	\item We perform an analysis on the application of  \gls{fsrcnn} and \gls{ircnn} in conjunction for tackling \gls{sisr} and image denoising tasks in order to restore \gls{lr} images with presence of different types of noise.
	\item We provide an evaluation comparing traditional and deep learning \gls{sisr} and image denoising methods on two popular testing datasets (Set5, Set14).
	\item We provide a Keras \cite{KERAS} implementation of \gls{fsrcnn} and \gls{ircnn} that can be used for both training and inference.
	\item We discuss possible future directions for \gls{sisr}+denoising research studies.
\end{itemize}

The remainder of this study is structured as follows: Section \ref{sec:background}, introduces the concepts of deep learning, \glspl{cnn}, \gls{sisr} and noise reduction. Section \ref{sec:deep_learning} summarizes the state-of-the-art \glspl{cnn} and other deep learning methods for \gls{sisr} and image denoising. Besides, it provides an overview of the structure of the two \glspl{cnn} used in this study: \gls{fsrcnn} and \gls{ircnn}. Section \ref{sec:experimental_design} describes the implementation details concerning the training, validation and test datasets, the data pre-processing and the training and inference strategies for the experiments that have been carried out. In Section \ref{sec:experiments}, we discuss the results of these experiments and provide a comparison with traditional methods for \gls{sisr} and denoising. Section \ref{sec:conclusions} presents the conclusions of this study and provides insight into possible future research directions.
	% !TeX spellcheck = en_US
\section{Background}\label{sec:background}

\subsection{Deep learning}
\gls{dl} is a set of machine learning techniques that aim at learning representations of data with several levels of abstraction by using models with multiple processing layers \cite{DL2}. Most of the deep learning architectures are based on \glspl{ann} due to their hierarchical property \cite{DL1} \cite{DBLP:DEEPSISR}.

These set of methods have been applied in multiple fields such as speech recognition, computer vision, genomics or natural language processing bringing important breakthroughs in many research areas. More specifically, \glspl{cnn} have had a relevant impact in the field of computer vision since they have achieved superior results in tasks like object recognition, video analysis, image classification or image restoration.

\subsection{Convolutional Neural Networks}
\gls{cnn} is a class of deep neural networks that has recently shown increasing popularity due to its success in natural language processing and computer vision fields.

\glspl{cnn} are different from others \glspl{ann} in the sense that \glspl{cnn} uses the convolution operation instead of matrix multiplication to propagate the data.

The architecture of a \glspl{cnn} varies depending on the task being performed, although they typically share:
\begin{itemize}
	\item An input layer that is a tensor with shape 
	$(n, r, c, d)$, where $n$ is the number of images to be processed, $r$ is the number of rows of pixels or image height, $c$ is the number of columns of pixels or image width, and $d$ is the image depth or number of channels.
	\item Multiple hidden layers that usually are convolutional layers that convolve their input with a set of filters producing a set of filtered images as a result that will be used for the next layer.
	\item Activation layers that apply an activation function to the result of the hidden layer. In \gls{cnn}, the most commonly used activation function is \gls{relu}, since it has demonstrated to make convergence faster \cite{RELU}.
\end{itemize}

Depending on the task being carried out, the architecture can also present:
\begin{itemize}
	\item Pooling layers that shrink the image stack produced by the convolution layers. These typically consist of filters of a given size that are used to downsize the resulting matrix of the convolution layer. There are different types of pooling layers depending on the used pooling function. These usually are: max pooling and average pooling.
	\item Fully connected layers 
\end{itemize}


\subsection{Single image super-resolution}

\subsection{Noise reduction}
	\section{Deep Learning for SISR and noise reduction}\label{sec:deep_learning}

\subsection{Deep Learning for Single Image Super-Resolution}
Recently, deep learning methods for \gls{sisr} have demonstrated promising performance and accuracy compared to conventional \gls{sisr} algorithms. Before \gls{srcnn} \cite{SRCNN} was presented, which is the pioneer work in the field, \cite{SISRBENCH}

\subsection{Deep Learning for Image Denoising}
\subsection{FSRCNN - Fast Super-Resolution CNN}
\subsection{IRCNN - Image Restoration CNN}
	\section{Experimental Design}\label{sec:experimental_design}

	% !TeX spellcheck = en_US
\section{Experiments}\label{sec:experiments}
In this section we present the results of all the experiments described above. 

All the evaluations are performed using Set5 \cite{SET5} and Set14 \cite{SET14} datasets.

First, all the images in Set5 and Set14 are downscaled using Lanczos interpolation with a scale factor of 2. Second, we apply one of the following noise types varying the noise parameters in order to enlarge the test set:
\begin{itemize}
	\item Gaussian noise is applied using $\mu=0$ and $\sigma = 0.05, \sigma = 0.10$ and $\sigma = 0.15$.
	\item Poisson noise is applied using input pixels values as means of Poisson distributions scaled up by $2^8$.
	\item Salt-and-pepper and uniform noise is applied using noise ratios of $0.05, 0.10, 0.15$ and $0.20$.
\end{itemize}

For the experiments that make use of median filter and wavelet denoiser, we apply the \textsc{matlab} functions \texttt{medfilt2()} and \texttt{wdenoise2()}.

In order to perform the evaluation of the quality metrics, the \textsc{matlab} functions \texttt{psnr()} and \texttt{ssim()} have been used.

\subsection{Experiments results}

\subsubsection{Experiment 1.1}
In this experiment, we evaluate \gls{fsrcnn} $+$ \gls{ircnn} on Set5 and Set14.

\begin{table}[]
	\centering
	\begin{tabular}{|l|l|r|r|r|r|}
		\hline
		\rowcolor[HTML]{EFEFEF} 
		\multicolumn{1}{|c|}{\cellcolor[HTML]{EFEFEF}\textbf{Noise}} & \textbf{Parameters} & \multicolumn{1}{c|}{\cellcolor[HTML]{EFEFEF}\textbf{Set5 \gls{psnr}}} & \multicolumn{1}{c|}{\cellcolor[HTML]{EFEFEF}\textbf{Set5 \gls{ssim}}} & \multicolumn{1}{c|}{\cellcolor[HTML]{EFEFEF}\textbf{Set14 \gls{psnr}}} & \multicolumn{1}{c|}{\cellcolor[HTML]{EFEFEF}\textbf{Set14 \gls{ssim}}} \\ \hline
		\rowcolor[HTML]{FFFFFF} 
		\cellcolor[HTML]{EFEFEF} & $\mu=0, \sigma=0.05$ & 26.6621 & 0.8639 & 25.1018 & 0.8065 \\
		\rowcolor[HTML]{EFEFEF} 
		\cellcolor[HTML]{EFEFEF} & $\mu=0, \sigma=0.10$ & 21.5502 & 0.7024 & 20.6851 & 0.6294 \\
		\rowcolor[HTML]{FFFFFF} 
		\multirow{-3}{*}{\cellcolor[HTML]{EFEFEF}Gaussian} & $\mu=0, \sigma=0.15$ & 18.4371 & 0.5669 & 17.7753 & 0.4954 \\
		\rowcolor[HTML]{EFEFEF} 
		Poisson & $peak=2^8$ & 27.9000 & 0.9074 & 25.9664 & 0.8401 \\
		\rowcolor[HTML]{FFFFFF} 
		\cellcolor[HTML]{EFEFEF} & $r=0.05$ & 22.1321 & 0.7678 & 21.1879 & 0.6994 \\
		\rowcolor[HTML]{EFEFEF} 
		\cellcolor[HTML]{EFEFEF} & $r=0.10$ & 18.6522 & 0.6218 & 18.0292 & 0.5487 \\
		\rowcolor[HTML]{FFFFFF} 
		\cellcolor[HTML]{EFEFEF} & $r=0.15$ & 16.4730 & 0.5141 & 16.0007 & 0.4431 \\
		\rowcolor[HTML]{EFEFEF} 
		\multirow{-4}{*}{\cellcolor[HTML]{EFEFEF}Salt-and-pepper} & $r=0.20$ & 14.9244 & 0.4320 & 14.5495 & 0.3670 \\
		\rowcolor[HTML]{FFFFFF} 
		\cellcolor[HTML]{EFEFEF} & $r=0.05$ & 24.3520 & 0.8135 & 23.4238 & 0.7688 \\
		\rowcolor[HTML]{EFEFEF} 
		\cellcolor[HTML]{EFEFEF} & $r=0.10$ & 20.9578 & 0.7032 & 20.5673 & 0.6542 \\
		\rowcolor[HTML]{FFFFFF} 
		\cellcolor[HTML]{EFEFEF} & $r=0.15$ & 18.7674 & 0.6155 & 18.6073 & 0.5642 \\
		\rowcolor[HTML]{EFEFEF} 
		\multirow{-4}{*}{\cellcolor[HTML]{EFEFEF}Uniform} & $r=0.20$ & 17.1090 & 0.5410 & 17.1459 & 0.4914 \\
		\rowcolor[HTML]{FFFFFF} 
		\textbf{All} &  & \textbf{20.6598} & \textbf{0.6708} & \textbf{19.9200} & \textbf{0.6090}\\\hline
	\end{tabular}
	\caption{\gls{psnr} and \gls{ssim} results for the Experiment 1.1}
	\label{tab:experiment11}
\end{table}

\begin{table}[]
	\centering
	\begin{tabular}{|l|l|r|r|r|r|}
		\hline
		\rowcolor[HTML]{EFEFEF} 
		\multicolumn{1}{|c|}{\cellcolor[HTML]{EFEFEF}\textbf{Noise}} & \textbf{Parameters} & \multicolumn{1}{c|}{\cellcolor[HTML]{EFEFEF}\textbf{Set5 \gls{psnr}}} & \multicolumn{1}{c|}{\cellcolor[HTML]{EFEFEF}\textbf{Set5 \gls{ssim}}} & \multicolumn{1}{c|}{\cellcolor[HTML]{EFEFEF}\textbf{Set14 \gls{psnr}}} & \multicolumn{1}{c|}{\cellcolor[HTML]{EFEFEF}\textbf{Set14 \gls{ssim}}} \\ \hline
		\rowcolor[HTML]{FFFFFF} 
		\cellcolor[HTML]{EFEFEF} & $\mu=0, \sigma=0.05$ & 29.4125 & 0.9055 & 27.1441 & 0.8704 \\
		\rowcolor[HTML]{EFEFEF} 
		\cellcolor[HTML]{EFEFEF} & $\mu=0, \sigma=0.10$ & 27.3907 & 0.8825 & 25.6060 & 0.8255 \\
		\rowcolor[HTML]{FFFFFF} 
		\multirow{-3}{*}{\cellcolor[HTML]{EFEFEF}Gaussian} & $\mu=0, \sigma=0.15$ & 26.0656 & 0.8709 & 24.4851 & 0.7896 \\
		\rowcolor[HTML]{EFEFEF} 
		Poisson & $peak=2^8$ & 30.1634 & 0.9314 & 27.5250 & 0.8855 \\
		\rowcolor[HTML]{FFFFFF} 
		\cellcolor[HTML]{EFEFEF} & $r=0.05$ & 32.3303 & 0.9585 & 28.8342 & 0.9176 \\
		\rowcolor[HTML]{EFEFEF} 
		\cellcolor[HTML]{EFEFEF} & $r=0.10$ & 31.8659 & 0.9498 & 28.5673 & 0.9119 \\
		\rowcolor[HTML]{FFFFFF} 
		\cellcolor[HTML]{EFEFEF} & $r=0.15$ & 31.3457 & 0.9453 & 28.2604 & 0.9067 \\
		\rowcolor[HTML]{EFEFEF} 
		\multirow{-4}{*}{\cellcolor[HTML]{EFEFEF}Salt-and-pepper} & $r=0.20$ & 30.6621 & 0.9395 & 27.8548 & 0.8990 \\
		\rowcolor[HTML]{FFFFFF} 
		\cellcolor[HTML]{EFEFEF} & $r=0.05$ & 32.2225 & 0.9559 & 28.7999 & 0.9160 \\
		\rowcolor[HTML]{EFEFEF} 
		\cellcolor[HTML]{EFEFEF} & $r=0.10$ & 31.5749 & 0.9459 & 28.4277 & 0.9089 \\
		\rowcolor[HTML]{FFFFFF} 
		\cellcolor[HTML]{EFEFEF} & $r=0.15$ & 30.8047 & 0.9390 & 27.9306 & 0.8999 \\
		\rowcolor[HTML]{EFEFEF} 
		\multirow{-4}{*}{\cellcolor[HTML]{EFEFEF}Uniform} & $r=0.20$ & 29.8574 & 0.9319 & 27.2938 & 0.8875 \\
		\rowcolor[HTML]{FFFFFF} 
		\textbf{All} &  & \textbf{30.3080} & \textbf{0.9297} & \textbf{27.5607} & \textbf{0.8849}\\\hline
	\end{tabular}
	\caption{\gls{psnr} and \gls{ssim} results for the Experiment 1.2}
	\label{tab:experiment12}
\end{table}

\begin{table}[]
	\centering
	\begin{tabular}{|l|l|r|r|r|r|}
		\hline
		\rowcolor[HTML]{EFEFEF} 
		\multicolumn{1}{|c|}{\cellcolor[HTML]{EFEFEF}\textbf{Noise}} & \textbf{Parameters} & \multicolumn{1}{c|}{\cellcolor[HTML]{EFEFEF}\textbf{Set5 \gls{psnr}}} & \multicolumn{1}{c|}{\cellcolor[HTML]{EFEFEF}\textbf{Set5 \gls{ssim}}} & \multicolumn{1}{c|}{\cellcolor[HTML]{EFEFEF}\textbf{Set14 \gls{psnr}}} & \multicolumn{1}{c|}{\cellcolor[HTML]{EFEFEF}\textbf{Set14 \gls{ssim}}} \\ \hline
		\rowcolor[HTML]{FFFFFF} 
		\cellcolor[HTML]{EFEFEF} & $\mu=0, \sigma=0.05$ & 29.1069 & 0.9188 & 26.7379 & 0.8608 \\
		\rowcolor[HTML]{EFEFEF} 
		\cellcolor[HTML]{EFEFEF} & $\mu=0, \sigma=0.10$ & 27.3585 & 0.8963 & 25.4602 & 0.8206 \\
		\rowcolor[HTML]{FFFFFF} 
		\multirow{-3}{*}{\cellcolor[HTML]{EFEFEF}Gaussian} & $\mu=0, \sigma=0.15$ & 26.1240 & 0.8824 & 24.4524 & 0.7874 \\
		\rowcolor[HTML]{EFEFEF} 
		Poisson & $peak=2^8$ & 29.7801 & 0.9419 & 27.0475 & 0.8742 \\
		\rowcolor[HTML]{FFFFFF} 
		\cellcolor[HTML]{EFEFEF} & $r=0.05$ & 31.2225 & 0.9585 & 27.9022 & 0.8994 \\
		\rowcolor[HTML]{EFEFEF} 
		\cellcolor[HTML]{EFEFEF} & $r=0.10$ & 30.9084 & 0.9538 & 27.7169 & 0.8953 \\
		\rowcolor[HTML]{FFFFFF} 
		\cellcolor[HTML]{EFEFEF} & $r=0.15$ & 30.5699 & 0.9503 & 27.5133 & 0.8908 \\
		\rowcolor[HTML]{EFEFEF} 
		\multirow{-4}{*}{\cellcolor[HTML]{EFEFEF}Salt-and-pepper} & $r=0.20$ & 30.1156 & 0.9454 & 27.2382 & 0.8842 \\
		\rowcolor[HTML]{FFFFFF} 
		\cellcolor[HTML]{EFEFEF} & $r=0.05$ & 31.1893 & 0.9577 & 27.8885 & 0.8988 \\
		\rowcolor[HTML]{EFEFEF} 
		\cellcolor[HTML]{EFEFEF} & $r=0.10$ & 30.7560 & 0.9519 & 27.6282 & 0.8933 \\
		\rowcolor[HTML]{FFFFFF} 
		\cellcolor[HTML]{EFEFEF} & $r=0.15$ & 30.2349 & 0.9463 & 27.2898 & 0.8858 \\
		\rowcolor[HTML]{EFEFEF} 
		\multirow{-4}{*}{\cellcolor[HTML]{EFEFEF}Uniform} & $r=0.20$ & 29.5822 & 0.9393 & 26.8459 & 0.8753 \\
		\rowcolor[HTML]{FFFFFF} 
		\textbf{All} &  & \textbf{29.7457} & \textbf{0.9369} & \textbf{26.9767} & \textbf{0.8722}
	\end{tabular}
	\caption{\gls{psnr} and \gls{ssim} results for the Experiment 2.1}
	\label{tab:experiment21}
\end{table}

\begin{table}[]
	\centering
	\begin{tabular}{|l|l|r|r|r|r|}
		\hline
		\rowcolor[HTML]{EFEFEF} 
		\multicolumn{1}{|c|}{\cellcolor[HTML]{EFEFEF}\textbf{Noise}} & \textbf{Parameters} & \multicolumn{1}{c|}{\cellcolor[HTML]{EFEFEF}\textbf{Set5 \gls{psnr}}} & \multicolumn{1}{c|}{\cellcolor[HTML]{EFEFEF}\textbf{Set5 \gls{ssim}}} & \multicolumn{1}{c|}{\cellcolor[HTML]{EFEFEF}\textbf{Set14 \gls{psnr}}} & \multicolumn{1}{c|}{\cellcolor[HTML]{EFEFEF}\textbf{Set14 \gls{ssim}}} \\ \hline
		\rowcolor[HTML]{FFFFFF} 
		\cellcolor[HTML]{EFEFEF} & $\mu=0, \sigma=0.05$ & 27.8830 & 0.9132 & 25.8943 & 0.8294 \\
		\rowcolor[HTML]{EFEFEF} 
		\cellcolor[HTML]{EFEFEF} & $\mu=0, \sigma=0.10$ & 25.0741 & 0.8574 & 23.6262 & 0.7520 \\
		\rowcolor[HTML]{FFFFFF} 
		\multirow{-3}{*}{\cellcolor[HTML]{EFEFEF}Gaussian} & $\mu=0, \sigma=0.15$ & 23.2414 & 0.8060 & 22.2485 & 0.7009 \\
		\rowcolor[HTML]{EFEFEF} 
		Poisson & $peak=2^8$ & 28.6318 & 0.9219 & 26.4398 & 0.8493 \\
		\rowcolor[HTML]{FFFFFF} 
		\cellcolor[HTML]{EFEFEF} & $r=0.05$ & 19.7492 & 0.6890 & 19.8054 & 0.6197 \\
		\rowcolor[HTML]{EFEFEF} 
		\cellcolor[HTML]{EFEFEF} & $r=0.10$ & 19.3873 & 0.6486 & 19.4879 & 0.5753 \\
		\rowcolor[HTML]{FFFFFF} 
		\cellcolor[HTML]{EFEFEF} & $r=0.15$ & 19.2579 & 0.6496 & 19.3416 & 0.5734 \\
		\rowcolor[HTML]{EFEFEF} 
		\multirow{-4}{*}{\cellcolor[HTML]{EFEFEF}Salt-and-pepper} & $r=0.20$ & 18.8251 & 0.6435 & 18.9623 & 0.5686 \\
		\rowcolor[HTML]{FFFFFF} 
		\cellcolor[HTML]{EFEFEF} & $r=0.05$ & 22.0364 & 0.7589 & 21.8637 & 0.7057 \\
		\rowcolor[HTML]{EFEFEF} 
		\cellcolor[HTML]{EFEFEF} & $r=0.10$ & 20.8710 & 0.7059 & 20.7881 & 0.6402 \\
		\rowcolor[HTML]{FFFFFF} 
		\cellcolor[HTML]{EFEFEF} & $r=0.15$ & 20.1205 & 0.6836 & 20.1378 & 0.6124 \\
		\rowcolor[HTML]{EFEFEF} 
		\multirow{-4}{*}{\cellcolor[HTML]{EFEFEF}Uniform} & $r=0.20$ & 19.3803 & 0.6659 & 19.5532 & 0.5959 \\
		\rowcolor[HTML]{FFFFFF} 
		\textbf{All} &  & \textbf{22.0382} & \textbf{0.7453} & \textbf{21.5124} & \textbf{0.6686}\\\hline
	\end{tabular}
	\caption{\gls{psnr} and \gls{ssim} results for the Experiment 2.2}
	\label{tab:experiment22}
\end{table}


\begin{table}[]
	\centering
	\begin{tabular}{|l|l|r|r|r|r|}
		\hline
		\rowcolor[HTML]{EFEFEF} 
		\multicolumn{1}{|c|}{\cellcolor[HTML]{EFEFEF}\textbf{Noise}} & \textbf{Parameters} & \multicolumn{1}{c|}{\cellcolor[HTML]{EFEFEF}\textbf{Set5 \gls{psnr}}} & \multicolumn{1}{c|}{\cellcolor[HTML]{EFEFEF}\textbf{Set5 \gls{ssim}}} & \multicolumn{1}{c|}{\cellcolor[HTML]{EFEFEF}\textbf{Set14 \gls{psnr}}} & \multicolumn{1}{c|}{\cellcolor[HTML]{EFEFEF}\textbf{Set14 \gls{ssim}}} \\ \hline
		\rowcolor[HTML]{FFFFFF} 
		\cellcolor[HTML]{EFEFEF} & $\mu=0, \sigma=0.05$ & 25.9872 & 0.8739 & 23.9859 & 0.7578 \\
		\rowcolor[HTML]{EFEFEF} 
		\cellcolor[HTML]{EFEFEF} & $\mu=0, \sigma=0.10$ & 23.9557 & 0.8044 & 22.5017 & 0.6852 \\
		\rowcolor[HTML]{FFFFFF} 
		\multirow{-3}{*}{\cellcolor[HTML]{EFEFEF}Gaussian} & $\mu=0, \sigma=0.15$ & 22.0852 & 0.7334 & 20.9973 & 0.6142 \\
		\rowcolor[HTML]{EFEFEF} 
		Poisson & $peak=2^8$ & 26.2772 & 0.8925 & 24.1578 & 0.7690 \\
		\rowcolor[HTML]{FFFFFF} 
		\cellcolor[HTML]{EFEFEF} & $r=0.05$ & 26.8715 & 0.9165 & 24.5762 & 0.7983 \\
		\rowcolor[HTML]{EFEFEF} 
		\cellcolor[HTML]{EFEFEF} & $r=0.10$ & 26.0928 & 0.9069 & 24.0433 & 0.7888 \\
		\rowcolor[HTML]{FFFFFF} 
		\cellcolor[HTML]{EFEFEF} & $r=0.15$ & 25.1191 & 0.8942 & 23.3845 & 0.7751 \\
		\rowcolor[HTML]{EFEFEF} 
		\multirow{-4}{*}{\cellcolor[HTML]{EFEFEF}Salt-and-pepper} & $r=0.20$ & 23.8783 & 0.8750 & 22.5070 & 0.7551 \\
		\rowcolor[HTML]{FFFFFF} 
		\cellcolor[HTML]{EFEFEF} & $r=0.05$ & 26.9496 & 0.9171 & 24.6094 & 0.7979 \\
		\rowcolor[HTML]{EFEFEF} 
		\cellcolor[HTML]{EFEFEF} & $r=0.10$ & 26.2939 & 0.9076 & 24.2091 & 0.7878 \\
		\rowcolor[HTML]{FFFFFF} 
		\cellcolor[HTML]{EFEFEF} & $r=0.15$ & 25.5000 & 0.8921 & 23.6985 & 0.7726 \\
		\rowcolor[HTML]{EFEFEF} 
		\multirow{-4}{*}{\cellcolor[HTML]{EFEFEF}Uniform} & $r=0.20$ & 24.4354 & 0.8651 & 23.0633 & 0.7503 \\
		\rowcolor[HTML]{FFFFFF} 
		\textbf{All} &  & \textbf{25.2872} & \textbf{0.8732} & \textbf{23.4778} & \textbf{0.7543}\\\hline
	\end{tabular}
	\caption{\gls{psnr} and \gls{ssim} results for the Experiment 2.3}
	\label{tab:experiment23}
\end{table}


\begin{table}[]
	\centering
	\begin{tabular}{|l|l|r|r|r|r|}
		\hline
		\rowcolor[HTML]{EFEFEF} 
		\multicolumn{1}{|c|}{\cellcolor[HTML]{EFEFEF}\textbf{Noise}} & \textbf{Parameters} & \multicolumn{1}{c|}{\cellcolor[HTML]{EFEFEF}\textbf{Set5 \gls{psnr}}} & \multicolumn{1}{c|}{\cellcolor[HTML]{EFEFEF}\textbf{Set5 \gls{ssim}}} & \multicolumn{1}{c|}{\cellcolor[HTML]{EFEFEF}\textbf{Set14 \gls{psnr}}} & \multicolumn{1}{c|}{\cellcolor[HTML]{EFEFEF}\textbf{Set14 \gls{ssim}}} \\ \hline
		\rowcolor[HTML]{FFFFFF} 
		\cellcolor[HTML]{EFEFEF} & $\mu=0, \sigma=0.05$ & 27.8760 & 0.9116 & 25.8165 & 0.8244 \\
		\rowcolor[HTML]{EFEFEF} 
		\cellcolor[HTML]{EFEFEF} & $\mu=0, \sigma=0.10$ & 25.2040 & 0.8553 & 23.7390 & 0.7520 \\
		\rowcolor[HTML]{FFFFFF} 
		\multirow{-3}{*}{\cellcolor[HTML]{EFEFEF}Gaussian} & $\mu=0, \sigma=0.15$ & 23.3125 & 0.8046 & 22.3406 & 0.7016 \\
		\rowcolor[HTML]{EFEFEF} 
		Poisson & $peak=2^8$ & 28.7267 & 0.9298 & 26.3338 & 0.8453 \\
		\rowcolor[HTML]{FFFFFF} 
		\cellcolor[HTML]{EFEFEF} & $r=0.05$ & 20.8489 & 0.7180 & 20.8049 & 0.6445 \\
		\rowcolor[HTML]{EFEFEF} 
		\cellcolor[HTML]{EFEFEF} & $r=0.10$ & 20.0257 & 0.6657 & 20.0924 & 0.5918 \\
		\rowcolor[HTML]{FFFFFF} 
		\cellcolor[HTML]{EFEFEF} & $r=0.15$ & 19.5516 & 0.6569 & 19.6361 & 0.5817 \\
		\rowcolor[HTML]{EFEFEF} 
		\multirow{-4}{*}{\cellcolor[HTML]{EFEFEF}Salt-and-pepper} & $r=0.20$ & 18.9040 & 0.6454 & 19.0571 & 0.5714 \\
		\rowcolor[HTML]{FFFFFF} 
		\cellcolor[HTML]{EFEFEF} & $r=0.05$ & 22.9494 & 0.7780 & 22.6777 & 0.7204 \\
		\rowcolor[HTML]{EFEFEF} 
		\cellcolor[HTML]{EFEFEF} & $r=0.10$ & 21.4818 & 0.7196 & 21.3892 & 0.6547 \\
		\rowcolor[HTML]{FFFFFF} 
		\cellcolor[HTML]{EFEFEF} & $r=0.15$ & 20.4685 & 0.6920 & 20.5239 & 0.6233 \\
		\rowcolor[HTML]{EFEFEF} 
		\multirow{-4}{*}{\cellcolor[HTML]{EFEFEF}Uniform} & $r=0.20$ & 19.5550 & 0.6708 & 19.7700 & 0.6028 \\
		\rowcolor[HTML]{FFFFFF} 
		\textbf{All} &  & \textbf{22.4087} & \textbf{0.7540} & \textbf{21.8485} & \textbf{0.6762}\\\hline
	\end{tabular}
	\caption{\gls{psnr} and \gls{ssim} results for the Experiment 3.1}
	\label{tab:experiment31}
\end{table}


\begin{table}[]
	\centering
	\begin{tabular}{|l|l|r|r|r|r|}
		\hline
		\rowcolor[HTML]{EFEFEF} 
		\multicolumn{1}{|c|}{\cellcolor[HTML]{EFEFEF}\textbf{Noise}} & \textbf{Parameters} & \multicolumn{1}{c|}{\cellcolor[HTML]{EFEFEF}\textbf{Set5 \gls{psnr}}} & \multicolumn{1}{c|}{\cellcolor[HTML]{EFEFEF}\textbf{Set5 \gls{ssim}}} & \multicolumn{1}{c|}{\cellcolor[HTML]{EFEFEF}\textbf{Set14 \gls{psnr}}} & \multicolumn{1}{c|}{\cellcolor[HTML]{EFEFEF}\textbf{Set14 \gls{ssim}}} \\ \hline
		\rowcolor[HTML]{FFFFFF} 
		\cellcolor[HTML]{EFEFEF} & $\mu=0, \sigma=0.05$ & 26.2755 & 0.8812 & 24.2264 & 0.7663 \\
		\rowcolor[HTML]{EFEFEF} 
		\cellcolor[HTML]{EFEFEF} & $\mu=0, \sigma=0.10$ & 24.2735 & 0.8134 & 22.8012 & 0.6979 \\
		\rowcolor[HTML]{FFFFFF} 
		\multirow{-3}{*}{\cellcolor[HTML]{EFEFEF}Gaussian} & $\mu=0, \sigma=0.15$ & 22.4366 & 0.7452 & 21.3403 & 0.6299 \\
		\rowcolor[HTML]{EFEFEF} 
		Poisson & $peak=2^8$ & 26.6459 & 0.9059 & 24.4262 & 0.7795 \\
		\rowcolor[HTML]{FFFFFF} 
		\cellcolor[HTML]{EFEFEF} & $r=0.05$ & 27.1121 & 0.9209 & 24.7449 & 0.8034 \\
		\rowcolor[HTML]{EFEFEF} 
		\cellcolor[HTML]{EFEFEF} & $r=0.10$ & 26.3782 & 0.9123 & 24.2646 & 0.7950 \\
		\rowcolor[HTML]{FFFFFF} 
		\cellcolor[HTML]{EFEFEF} & $r=0.15$ & 25.4810 & 0.9009 & 23.6647 & 0.7829 \\
		\rowcolor[HTML]{EFEFEF} 
		\multirow{-4}{*}{\cellcolor[HTML]{EFEFEF}Salt-and-pepper} & $r=0.20$ & 24.3109 & 0.8836 & 22.8559 & 0.7650 \\
		\rowcolor[HTML]{FFFFFF} 
		\cellcolor[HTML]{EFEFEF} & $r=0.05$ & 27.1583 & 0.9208 & 24.7666 & 0.8026 \\
		\rowcolor[HTML]{EFEFEF} 
		\cellcolor[HTML]{EFEFEF} & $r=0.10$ & 26.5295 & 0.9116 & 24.3931 & 0.7933 \\
		\rowcolor[HTML]{FFFFFF} 
		\cellcolor[HTML]{EFEFEF} & $r=0.15$ & 25.7746 & 0.8967 & 23.9241 & 0.7793 \\
		\rowcolor[HTML]{EFEFEF} 
		\multirow{-4}{*}{\cellcolor[HTML]{EFEFEF}Uniform} & $r=0.20$ & 24.7671 & 0.8715 & 23.3377 & 0.7590 \\
		\rowcolor[HTML]{FFFFFF} 
		\textbf{All} &  & \textbf{25.5953} & \textbf{0.8803} & \textbf{23.7288} & \textbf{0.7628}\\\hline
	\end{tabular}
	\caption{\gls{psnr} and \gls{ssim} results for the Experiment 3.2}
	\label{tab:experiment32}
\end{table}




\begin{table}[]
	\centering
	\begin{tabular}{|l|r|r|r|r|}
		\hline
		\rowcolor[HTML]{EFEFEF}
		 \multicolumn{1}{|c}{\cellcolor[HTML]{EFEFEF}\textbf{Method}} &
		 \multicolumn{1}{c|}{\cellcolor[HTML]{EFEFEF}\textbf{Set5 \gls{psnr}}} & \multicolumn{1}{c|}{\cellcolor[HTML]{EFEFEF}\textbf{Set5 \gls{ssim}}} & \multicolumn{1}{c|}{\cellcolor[HTML]{EFEFEF}\textbf{Set14 \gls{psnr}}} & \multicolumn{1}{c|}{\cellcolor[HTML]{EFEFEF}\textbf{Set14 \gls{ssim}}} \\ \hline
		\rowcolor[HTML]{FFFFFF} 
		\gls{fsrcnn} & 33.9226 & 0.9733 & 29.8034 & 0.9304\\
		\rowcolor[HTML]{EFEFEF} 
		Bicubic Interpolation & 32.2347 & 0.9677 & 28.6030 & 0.9115\\\hline
		\textbf{\gls{fsrcnn} - Bicubic} & \textbf{+1.6879} & \textbf{+0.0056} & \textbf{+1.2004} & \textbf{+0.0189}\\\hline
	\end{tabular}
	\caption{\gls{psnr} and \gls{ssim} results for \gls{fsrcnn} vs Bicubic Interpolation}
	\label{tab:experiment}
\end{table}


\subsection{Results summary}
	% !TeX spellcheck = en_US
\section{Conclusions and further developments} \label{sec:conclusions}

\gls{sisr} and image denoising are two important low-level computer vision tasks with multiple applications in several fields. In the last few decades, \glspl{cnn} have shown promising results while performing such kinds of restoration tasks, receiving increasing attention from researches.

In this study, we have analyzed the results of the combined application of two well known \glspl{cnn} for \gls{sisr} and image denoising: \gls{fsrcnn} and \gls{ircnn}, respectively.
Our results show that the order in which these algorithms are applied has a direct effect in their combined performance: if the \gls{sisr} algorithm is applied first, those pixels affected by the image noise will be convolved with the good information, producing low quality restoration results. Instead, if the denoiser is applied first, the \gls{sisr} will manage to super-resolve the \gls{lr} image.

For this analysis, both \glspl{cnn} have been trained using the same images from BSD200 \cite{BSDS}, General100 \cite{FSRCNN} and T91 \cite{T91} datasets. A Keras \cite{KERAS} implementation is also provided with this study.

Furthermore, we have carried out a set of experiments comparing the performance of these deep learning methods to traditional algorithms for the \gls{sisr} and denoising tasks. In all the cases, the combination of \gls{ircnn} and \gls{fsrcnn} shows better performance than when bicubic interpolation is used for \gls{sisr} and wavelet denoising or median filtering is applied for denoising.

Although the results of this study show that the deep learning approach using \gls{ircnn} + \gls{fsrcnn} overperforms the selected traditional methods, there remain several open questions for future research works.

First of all, most of the research studies simulate the \gls{lr} images by modeling the degradation with algorithms like bicubic or Lanczos interpolation and Gaussian noise, whereas in real scenarios the \gls{lr} images may present a different distribution, and therefore the learned mappings might not be useful for these cases.

Furthermore, a more extensive comparison with other traditional methods for \gls{sisr} and image denoising could be addressed, including algorithms such as total variation denoising.

On the other hand, as mentioned earlier in this study, image denoisers can also be used for \gls{sisr} if the noise is modeled as the difference between the \gls{hr} image and the bicubic upsampling
of the \gls{lr} image. A possible research direction could be to investigate a network that solves \gls{sisr} and denoising simultaneously using multiple types of noise.

	
	\printbibliography
	\printnoidxglossary[type=acronym]
	\printacronyms

\end{document}
