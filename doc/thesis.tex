\documentclass[runningheads]{llncs}

% Full page margins
\usepackage{fullpage}

% For language
\usepackage[english]{babel}
\usepackage[utf8]{inputenc}

% For figures
\usepackage{graphicx}

% Standard colors
\usepackage[dvipsnames]{xcolor}

% For links
\usepackage{hyperref}
\usepackage{cleveref}
\hypersetup {
	linkcolor  = MidnightBlue,
	citecolor  = MidnightBlue,
	urlcolor   = MidnightBlue,
	colorlinks = true,
}

% For bibliography
\usepackage[backend=biber]{biblatex}
\addbibresource{references.bib}

% TODO remove
\usepackage{lipsum}

\begin{document}

	% Title
	\title{Single Image Super-Resolution and Denoising using Convolutional Neural Networks}
	\author{Guillermo Ruiz Álvarez}
	\institute{University of Málaga - \email{grabmct@uma.es}}
	
	\maketitle
	
	% TODO finish abstract
	\abstract{Single image super-resolution and image denoising are two common tasks in low-level computer vision applications. Single image super-resolution is a classical problem that aims at recovering a high resolution image from a low resolution version of the same image. Image denoising is the process of estimating the clean and original version of an image from a noisy input. In this study we analyse the behaviour of two well-known methods for single image super-resolution and denoising, namely SRCNN [TODO reference] and IRCNN [TODO reference], applied in different order to noisy, low resolution inputs in order to evaluate their combined capability to super-resolve low resolution images with presence of Gaussian noise, Poisson noise and salt-and-pepper noise. As a result we conclude that [TODO] }
	
	% TODO keywords
	\keywords{Single Image Super-Resolution, Image Denoising, Image Restoration, Convolutional Neural Networks.}\\~
	
	\textbf{Supervisors:} Ezequiel López Rubio, Rafael Marcos Luque Baena. 
	
	\section{Introduction}
\lipsum[1] \cite{pSRCNN} \cite{SRCNN} \cite{FSRCNN} \cite{IRCNN}
	\section{Background}
\lipsum[1]
	\section{Experiments}
\lipsum[1]
	\section{Conclusions}
\lipsum[1]
	
	\printbibliography

\end{document}