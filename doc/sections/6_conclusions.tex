% !TeX spellcheck = en_US
\section{Conclusions and further developments} \label{sec:conclusions}

\gls{sisr} and image denoising are two important low-level computer vision tasks with multiple applications in several fields. In the last few decades, \glspl{cnn} have shown promising results while performing such kinds of restoration tasks, receiving increasing attention from researches.

In this study, we have analyzed the results of the combined application of two well known \glspl{cnn} for \gls{sisr} and image denoising: \gls{fsrcnn} and \gls{ircnn}, respectively.
Our results show that the order in which these algorithms are applied has a direct effect in their combined performance: if the \gls{sisr} algorithm is applied first, those pixels affected by the image noise will be convolved with the good information, producing low quality restoration results. Instead, if the denoiser is applied first, the \gls{sisr} will manage to super-resolve the \gls{lr} image.

For this analysis, both \glspl{cnn} have been trained using the same images from BSD200 \cite{BSDS}, General100 \cite{FSRCNN} and T91 \cite{T91} datasets. A Keras \cite{KERAS} implementation is also provided with this study.

Furthermore, we have carried out a set of experiments comparing the performance of these deep learning methods to traditional algorithms for the \gls{sisr} and denoising tasks. In all the cases, the combination of \gls{ircnn} and \gls{fsrcnn} shows better performance than when bicubic interpolation is used for \gls{sisr} and wavelet denoising or median filtering is applied for denoising.

Although the results of this study show that the deep learning approach using \gls{ircnn} + \gls{fsrcnn} overperforms the selected traditional methods, there remain several open questions for future research works.

First of all, most of the research studies simulate the \gls{lr} images by modeling the degradation with algorithms like bicubic or Lanczos interpolation and Gaussian noise, whereas in real scenarios the \gls{lr} images may present a different distribution, and therefore the learned mappings might not be useful for these cases.

Furthermore, a more extensive comparison with other traditional methods for \gls{sisr} and image denoising could be addressed, including algorithms such as total variation denoising.

On the other hand, as mentioned earlier in this study, image denoisers can also be used for \gls{sisr} if the noise is modeled as the difference between the \gls{hr} image and the bicubic upsampling
of the \gls{lr} image. A possible research direction could be to investigate a network that solves \gls{sisr} and denoising simultaneously using multiple types of noise.
